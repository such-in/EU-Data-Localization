\section{Discussion}
In this section we discuss our findings on data localization compliance
in the EU.

\subsection{Causes for Servers/Trackers in Non-Adequate Destinations}
Our method does not reveal the causes for web servers or trackers to be located in 
non-adequate countries, which is potentially unlawful in the EU.
We speculate about some potential causes here.
First, the content provider is intent in serving EU users from the EU (or an adequate country),
but in the presence of temporary server or router outages, it directs the user request to a backup
location elsewhere.
Second, the content owners may not know that the user is located in the EU,
as Internet protocols were not designed with physical nor jurisdictional constraints.
For instance, \textit{vk.com} might mistakenly infer that a user is located within Russia, thus
directing them to their servers there.
Third, in some instances, performance considerations may trump legal compliance.
For instance, users in Romania might experience much higher latency if their
content is served from a server in, say, Belgium, rather than Turkey, which is much closer.

\subsection{US as Destination}
The US is the most frequent non-adequate destination in our data,
observed from all but two EU countries in the sample.
The US is, of course, a key provider of cloud services,
and the home base of most of the world's largest content providers.
Partly given the amount of economic activity spurred by data transfers
between the US and the EU, the executive branches of both jurisdictions are intent on generating a
framework that broadly authorizes such transfers~\cite{EUUSData30:online,FACTSHEE89:online}. 
Such a framework is not guaranteed to stand up in the EU courts as previous attempts at
authorizing transfers from the EU to the US have failed in the EU's legal system~\cite{ThirdTim84:online}.
Whether the latest
EU-US framework will pass muster under future judicial review remains an open question
according to legal experts~\cite{ceps}.

\subsection{Regulatory Considerations}
The wide variety of trackers observed, loaded from an also varied
set of popular sites in each country, may pose challenges of scale for case-by-case regimes
such as the Standard Contractual Clauses (SCCs)~\cite{sccs},
particularly when auditing compliance. 
These have been applied to US companies~\cite{leglobal}
and are also in consideration~\cite{sccs} for data transfers to countries in 
Southeast Asia (ASEAN). We observed multiple ASEAN member countries~\cite{asean}
as destinations in our data: Thailand, Singapore and Malaysia. While
they are relatively uncommon destinations in our sample, EU-ASEAN
agreements~\cite{sccs} might change that in the future.

\subsection{News Sites}
The concentration we observed, where trackers in non-adequate countries
are primarily loaded by news sites, provides a potential opportunity
when designing data localization auditing frameworks in other jurisdictions
outside the EU: the websites in this
category should be evaluated first. This prioritization might 
be particularly useful in pilot evaluations of compliance with newly introduced
data localization requirements, or in environments where data collection is constrained
for instance by available bandwidth or where electric supply is less reliable.

\subsection{Regional Differences}
Perhaps the most surprising result in our study is the
differing rates of compliance across EU regions. While
certainly there are cultural and economic differences between
them, in theory it is not more challenging for a content provider
to deliver content from within Europe: there is ample available infrastructure
within the continent. However the disparities in 
compliance with data localization principles for trackers were significant. 
This uneven compliance with regulation intended to 
protect user privacy poses questions in equity and fairness:
the wealthier regions of the EU (Northern and Western) might 
be able to provide more uniform protections to their users than the 
less affluent regions in the south and east. Extrapolating this to
other jurisdictions who may be considering, or have recently implemented
data localization requirements, these equity and fairness considerations
are worth examining empirically. Empirical audits as the one we have presented in 
this study can reveal these issues 
with the greatest clarity.

\subsection{Russia and Finland}
The case of \textit{vk.com}, as accessed in Finland, presents a particular challenge
for data privacy. The company's CEO has been sanctioned by the EU in connection 
with the Ukraine conflict~\cite{vksanction}; the company 
is also state-owned (by Russia)~\cite{russiavk}. However,
since the services provided by VK are popular among Russian speakers,
which are a substantial portion of the population of many Eastern EU countries
and Finland, an outright ban of the service would negatively impact
these groups. We do find that \textit{vk.com} loads third party trackers in Russia,
creating a potential geopolitical challenge in addition to the usual potential privacy harms.

\section{Server Geolocation Validation}
\label{sec:testbed}

To validate our server geolocation method, we build an experiment 
based on two sets of server IPs with known location.
Each of these sets is based either in a non-adequate country, the 
US,~\footnote{At the time of our main data collection. The validation experiment 
was conducted in Feb. 2025, when the US had already received an adequacy decision.~\cite{latham}}
or an adequate country, an EU member state.
To conduct this validation experiment, we followed the four steps 
of the methodology defined in \S~\ref{sec:servergeolocation} to perform 
server IP geolocation: RIPE IPmap geolocation, source-based constraints, 
destination-based constraints, and reverse DNS (rDNS).

The results of this experiment are shown in Tab.~\ref{tab:testbedgeo}.
In the following subsections, we describe this validation experiment.
We also infer rates of true/false positives and precision of our method.
Note that, as discussed in the limitations section (\S~\ref{sec:falsepos}), the 
rate of false positives might be different in our main results.




\subsection{Server Testbed in the US}
We have administrative access 
to a US testbed, CloudLab ~\cite{CloudLab}, and thus to ground truth mapping between server IP and country of operation. 
We conducted experiments using 200 such servers from a testbed that are
distributed across two different locations (L1 and L2), both in the US.
These servers are the destination IPs for this experiment. 
Further, we selected five source countries from 
the European Union (EU). We included the largest EU country, France (by land area), 
and four additional countries chosen randomly to represent different regions of the EU: 
Poland from Eastern Europe, Germany from Western Europe, Ireland from Northern Europe, 
and Spain from Southern Europe.
For each source country, we randomly selected 40 server IPs as destinations across 
the two testbed locations (L1 and L2). This results in a total of 200 destination IP-source
country pairs (DSCPs). 

The upper half of Tab.~\ref{tab:testbedgeo} shows the results of this
experiment. In the rest of this paragraph, we describe these 
findings in more detail. 
We found that RIPE IPmap did not provide country-level geolocation 
for three server IPs. %
In the source-based traceroutes, five server IPs failed to respond, and 20 did not meet the 90\% 
latency threshold described in \S~\ref{sec:source-based}. 
We further excluded two server IPs
using destination-based constraints. %
Finally, when we analyzed rDNS data, 
three of these servers had clues indicating they were not located in the city identified by RIPE IPmap,
but they were still located in the US. As our method is concerned with
country-level geolocation, we keep the server geolocation unchnaged from previous steps.
In summary, our final dataset correctly geolocated 170 DSCPs 
as being in the US,
and discarded 30 DSCPs due to the constraints described above.








\vspace{-10mm}
\subsection{AWS Servers in the EU}

To validate our method for adequate server IPs, 
we conducted an experiment using 1,000 IPs in the EU 
advertised by Amazon Web Services~\cite{aws_ip_ranges}. 
We selected these IPs by intersecting the AWS-published IP ranges
with the ISI IP Hitlist,~\cite{ant_hitlist} which estimates the likelihood
that a server will respond to network measurements.
The IPs we select have a score of 99 on the Hitlist. 
We take this step in an attempt to maximize response rates to our measurements 
since, unlike on the US testbed, 
we have no direct knowledge of whether the AWS IPs are currently
both routed and in use. This is also why we increase the 
number of server IPs in the EU relative to the US experiments in the previous
subsection.

Similarly to our treatment of the US testbed, we divide the server IPs 
into five groups corresponding to the same five 
EU source countries. 
The lower half of Tab.~\ref{tab:testbedgeo} shows the results of this experiment.

From these 1,000 DSCPs, we find that RIPE IPMap 
does not assign a country in 36 cases, and erroneously geolocates 
270 DSCPs outside of the EU in non-adequate countries: 
one in Pakistan, the rest in the US. 
After conducting both our source- and destination-based measurements, 
and applying speed of light constraints, these 270 DSCPs are all discarded.
At this stage, 27 DSCPs are correctly labeled as being in an (adequate)
EU country.
rDNS confirms this assertion in all 27 cases.



\begin{table}
  \caption{DSCP Geolocation for server validation experiment. 
ST/DT: Source/Destination Traceroutes. 
TB: Server Testbed. *No hostname/no geohint.}
  \label{tab:testbedgeo}
  \begin{tabular}{ccccc}
    \toprule
    \textbf{Method} & \textbf{DSCPs} & \textbf{Unres-} & \textbf{Adequate} & \textbf{Non-}\\
     &  & \textbf{ponsive} &  & \textbf{adequate} \\
    \toprule
    ST: US TB & 197  & 5 & 0 & 172 \\
    DT: US TB & 172 & 0 & 0& 170 \\
    rDNS: US TB & 170 & 99/0* & 0 & 170 \\
    \midrule
    ST: AWS & 964  & 910 & 50  & 0 \\
    DT: AWS & 50  & 21 & 27  & 0 \\
    rDNS: AWS & 27  & 0/0* & 27  & 0 \\
  \bottomrule
\end{tabular}
\end{table}


\subsection{True/False Positives and Negatives, Precision}
In this subsection, we describe the rates of
true/false positives and negatives,
along with precision, that result
from our validation experiments.

\textbf{False positives and true negatives}. 
The outcome of the AWS experiment is either a false positive,
where a server 
in an adequate country is incorrectly inferred as being in a
non-adequate country; or a true negative,
where a server in an adequate country is correctly labeled as being so.

Given the results of the AWS validation experiment, 
we infer \textit{no false positives}.
This is because all servers initially labeled as being outside the 
EU by RIPE IPMap were correctly discarded---assigned a negative label---by 
subsequent steps in our method. 
Note that our source- and destination-based filters 
aggressively discard DSCPs because they are
either unresponsive or the latency fails our speed of light constraints.
Thus, after all filters are applied, 
the \textit{true negative} rate is 100\%.

In aggregate, these results suggest that the coverage of our method is limited,
because it excludes servers that may actually be located 
in a non-adequate country. 
Simultaneously, our method reduces the likelihood of false positives. 
Thus, in this experiment, our method is working as intended.


\textbf{True positives and false negatives}. Recall that the US was a 
non-adequate destination at the time of our main data collection,
so all servers in the US testbed
have a non-adequate country as their ground truth location; %
thus, all DSCPs in the US testbed validation experiment are either
correctly allocated to the US, a true positive, or discarded, a false negative.
Our method correctly identified 170 DSCPs as being in the US,
of 200 DSCPs known to be in the US.
Therefore, the \textit{true positive} rate of our method in this experiment is 85\%,
and the \textit{false negative} rate is 15\%.

\textbf{Precision}. From the aforementioned results,
in particular given the absence of false positives,
the inferred \textit{precision} of our method 
in the validation experiments is 1.0.



\section{Proxy Location Validation}
\label{sec:proxyval}
We conduct an experiment to investigate whether
BrightData's claims about requests being routed through 
an AS-Country Pair are accurate. To this
end, we set up a web server at Northeastern University and
send HTTP requests through BrightData from each 
ASCP. All of the requests this server received were IPv4,
and we take steps to preserve the privacy of BrightData users
(who host the proxies in their own devices) by recording only the 
/24 subnet from which we received the request to our university server. 
We then compare the country and AS claimed by 
BrightData with those identified by geolocation database 
Maxmind~\cite{maxmindgeoloc}.
We fetch the AS and country for every IP in the /24 prefix through Maxmind.

We find that BrightData seems to be almost always routing
requests through the ASCP they claim.
Of the 2,319 valid requests received by this server from BrightData,
all but five are accurate. Thus, 2,314 requests have an IP that is 
part of a /24 prefix entirely present in the same ASCP according
to Maxmind.
The five exceptions include two where the country does not match (but the AS does), two where the AS does not match
(but the country does), and one where neither AS nor country are a match.
Therefore, we conclude that BrightData is an appropriate proxy to use for the 
purposes of routing requests through a specific AS in a given country.

We acknowledge that geolocation databases are prone to errors. However,
since we are working at the country level granularity, 
these errors are less common~\cite{10.1007/978-3-030-98785-5_6,10.1145/1971162.1971171}.
Of course, it is possible that both BrightData and Maxmind are often both incorrect and in agreement about
the ASCP where a user is located, but we argue that this is a remote possibility.

\section{Limitations}
In this section, we describe the main limitations of our approach. 

\subsection{Domain Exclusions}
\label{sec:domexc}
As described in \S~\ref{sec:domains}, our framework excludes
both Google-owned and adult websites as initial domains.
This exclusion is caused by BrightData rules. Thus,
our method is not able to study
these two groups of domains, particularly as
initial sites. For Google domains, this limitation
does not apply when they are loaded as third-parties by
other websites that are not Google owned.

\subsection{Potential for False Positives}
\label{sec:falsepos}
Since we do not have access to ground truth 
on physical server location, the rate of false positives in our
results is not known. 
For instance, there could be interactions between the 
server response rate to traceroutes and their geographic
location, which would impact our findings.
While we conducted a validation experiment
using servers with known location (\S~\ref{sec:testbed}),
it is still possible that the servers loaded by popular EU sites,
\textit{i.e.}, the servers we study in our main results section,
respond differently to our measurements than the servers in the
validation experiments. Furthermore, 
the EU servers in the validation experiment may treat 
traceroutes differently from other traffic,
potentially impacting our latency-based geolocation techniques.
Finally, the scale of our validation experiment is smaller than
the experimental setup in our main results,
which may impact the former's generalizability.
Therefore, the false positive rates in the validation 
experiment %
might differ from those in our main results.
Due to all the aforementioned factors, 
the lack of information regarding false positive rates and
precision in our main results is a limitation of this study.



\subsection{Alternative CDN Nodes}
A key limitation of our study 
is that we do not collect alternative server locations
for each destination, for instance available CDN nodes
~\cite{10.1145/1159913.1159962} from each ASCP.
This additional CDN-node data
would  potentially reveal whether the content providers are making their best effort to
comply with GDPR's data localization policies, \textit{i.e.}, they are
picking compliant servers even when they offer worse performance.
Alternatively, the content providers could instead primarily be selecting these servers
based on performance attributes such as latency.
Thus, our study is not able to distinguish between 
server placement that is based on performance constraints or
GDPR data localization compliance.

\subsection{Inference of GDPR Violations}

Our study is not able to make final determinations on GDPR violations. 
This is because legal exemptions may exist to mitigate the instances
where we have identified servers in non-adequate countries.
Further, a ruling on any potential GDPR violation would require additional
context on the specific data that was transferred and any legal contracts
in place between various entities, including privacy regulators~\cite{sccs}.
Ultimately, as such a determination of a GDPR violation would likely 
take place in a judicial context and be subject
to considerations beyond the physical location of a server. 

\subsection{Framework Applicability in Other Regions}
Our framework has limited applicability beyond the EU due to measurement
infrastructure density. Previous work has found that 
RIPE Atlas has more density of
deployment in Europe compared to other regions~\cite{ripeatlasbias,10198985}.
While geographic bias in the BrightData platform has not been as rigurously
quantified, even a cursory look at their top proxy locations reveals potential bias,
with the US and India having a similar number of IPs available even though the
latter is considerably larger in terms of Internet users.~\cite{topproxylocations}




\section{Related Work}
\label{sec:related}

Iordanou et al. ~\cite{10.1145/3278532.3278561} studies the
cross-border web tracking that targets users in the EU,
specifically the geographic scope of these data flows.
There are some similarities between this study and our work.
As in our study,
the previous work leverages large-scale Internet measurements,
partially launched from real end-user devices,
to study the location of servers responding to requests in the EU.
Both the related work and our study conclude that 
most traffic stays within the EU (or, in our case,
in the EU and adequate third countries); and
both also rely on RIPE IPmap for server geolocation, though our
framework also collects both source- and destination-based
measurements, and rDNS data, to validate each individual server inferred to be 
in a non-adequate country.

However, this previous work ~\cite{10.1145/3278532.3278561} differs from ours
in three key ways.
First, their focus is entirely on 
tracking servers, as they are interested in tracking flows, whereas we
investigate both general-purpose servers as well as tracking servers,
as we are instead interested in auditing data localization.
Second, the related work's
focus is on maximizing coverage of potential \textit{tracking flows},
whereas our study is focused on obtaining a representative
sample of \textit{source networks} in the EU. Third,
Iordanou et al. launch measurements from a browser
extension deployed to 183 EU users over a longer
period of several months, and intersect the
tracking servers there observed in ISP NetFlow data from
four large networks in three countries, whereas we rely 
on a point-in-time collection from a proxy deployed to more than 1,000 networks
in 20 countries.


Previous work has also tackled the issue of identifying cross-border data 
transfers. %
Guam{\'a}n et al.~\cite{9328756} study
the geographic spread of data flows originating from Android apps and evaluates
whether these flows comply with both the developer's privacy policies as
well as the GDPR. 
Razaghpanah et al. ~\cite{DBLP:conf/ndss/RazaghpanahNVSA18}
analyze tracking by mobile apps and whether these flows comply with
EU regulations, including an analysis of non-EU server locations.
Similarly, Nan et al. ~\cite{285453}
apply static analysis techniques to IoT companion apps (the smartphone apps needed to operate 
most IoT devices) and reveal data exposure caused by these devices.
Finally, Urban et al. ~\cite{10.1145/3366423.3380203} study third-party services in the
Web to generate a tree of dependencies that represents which additional services
are loaded by each initial third party contacted by a website.
While not a central focus of the latter two studies,
the authors do investigate the country where third-party servers that receive IoT/Web data (respectively) are located.
We note that all the studies cited in this paragraph rely on a geolocation
database (prone to inaccuracies ~\cite{10.1145/1971162.1971171, 10.1007/978-3-030-98785-5_6}) to pin the location of servers,
a method which often overstates the prevalence of servers located in the US~\cite{10.1145/3278532.3278561}, 
without conducting additional verification using Internet measurements,
as we have done in our work. Thus, we argue that our framework is more applicable
for auditing compliance with data localization requirements.



\section{Conclusion and Future Work}
A key component of the GDPR is the data localization regulation.
However, to date there was not a method to audit compliance
with this requirement at continental scale.
Our method fills this gap and provides a framework 
for empirically auditing compliance
with data localization principles and laws, and specifically investigate
such compliance for known trackers, which pose a greater privacy risk.
To accomplish this, we collect both browser-based data, the websites and 
domains loaded while browsing popular sites, and network-based data,
the servers that respond to such requests. We find that 
data localization requirements are broadly complied with in the EU,
though there are meaningful exceptions, which are more prevalent 
some regions. %

In the future, we plan on investigating the causes of high compliance rates,
disambiguating between legal compliance and performance prioritization by
web content companies. We further plan on applying this framework to regions
beyond the EU. In these regions, there are regulations that differ from the GDPR
on data localization to varying extents, providing opportunities for comparative
analyses of policy effectiveness with regards to data localization.

\section{Acknowledgements}
We thank the anonymous reviewers and the revision editor for their
valuable feedback. We are grateful to BrightData and RIPE Atlas
for allowing us to collect data using their platforms. 
This research was funded in part by the 
US National Science Foundation (NSF), Grants No. CNS 1955227 and CNS 2402963.
Author Gamero-Garrido was supported in part by Northeastern University's 
Future Faculty Fellowship and the Ford Foundation Postdoctoral Fellowship.
