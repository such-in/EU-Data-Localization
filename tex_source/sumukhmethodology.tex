





\subsection{Source-Based Measurements}
\label{sec:source-based}


We first obtain a preliminary assessment of where the server is located using RIPE IPmap,
step (6) of Fig.~\ref{fig:methodoverview}. %
This assessment is preliminary since even more accurate geolocation databases can err at the country level.
The passive inference %
provides us with a list of candidate server IPs that
might be located in a non-adequate country.
In this and further subsections, we aim to
identify instances of erroneous inference by IPMap;
in particular, we identify those where the server IP
is located in the EU or an adequate country but that were inferred by IPMap
as being in a non-adequate country. In other words, we look to identify false positives
in our identification of potential GDPR violations.

Our initial step to accomplish this goal
launching traceroutes towards the servers
(hostnames) previously inferred as being located in a non-adequate country,
step (5) of Fig.~\ref{fig:methodoverview}.
Note that this approach ensures that the DNS resolution is conducted 
on the same ASCP as the source of BrightData measurements.
Then, we identify candidate servers that may be located in non-adequate countries since
both the traceroute latency and IPMap support that inferred location.
This data collection took place in August of 2022.

Specifically, in this step we look for latency between the EU-based
RIPE Atlas probe and the destination server (hostname)
that is consistent with latency statistics published by Verizon~\cite{MonthlyI86:online};
this is depicted in step (6) of Fig.~\ref{fig:methodoverview}.
Since Verizon does not publish latency data between Latin America and the EU, we
rely on wondernetwork.com/pings~\cite{GlobalPi56:online} for these destinations.
In both cases, we impose a requirement that the observed latency is at least 90\%
of the average for that destination. These thresholds vary widely depending on 
the non-EU and non-adequate destination: Europe (13ms), US (65ms), EMEA region (78ms), 
Asia-Pacific region (106ms), Latin America (113-166ms depending on the country).

We launch 9,905 traceroutes towards servers in non-adequate countries (as per IPMap).\footnote{We launched 
traceroutes also towards servers in EU and adequate countries; we do not analyze these traceroutes here,
as they are unlikely contribute information on destinations in non-adequate countries.} 
In 9,296 of these cases, we analyze the traceroute latency to the last hop, 
subtracting the latency from the first hop
when possible to avoid increased latency in the last mile, \textit{e.g.}, due to WiFi.
In an additional 451 instances we use last hop latency. 
We exclude 158 traces due to either an unresponsive last hop 
(28) or latency that is higher to the first hop than the last (130).
In 8,488 traceroutes we observe latency that is below our threshold for that destination,
and we exclude these from further investigation. We are left with 1,259 traceroutes
that suggest that a server is located in a non-adequate country;
recall that the European Commission has designated a number of
non-EU member countries as being ``adequate'' for the purposes of
the GDPR's data localization requirement.~\cite{Adequacy38:online}


\subsection{Destination-Based Measurements}
\label{sec:destination-based}
To further confirm that responding servers are located in non-adequate countries, we 
collect additional evidence from RIPE Atlas probes located in those same countries.
This data collection took place in November/December of 2022.
We then use speed of light (subsequently denoted by $c$) constraints to 
discard likely erroneous geolocation inferences by IPMap.
Our goal is to remove as many false positives as possible from our experiments using empirical network data. 
However, there may still be false positives in our results; 
we describe this limitation of our work in \S~\ref{sec:falsepos}.

We launch traceroutes from RIPE Atlas probes located in the same non-adequate country
where the server was inferred to be located by IPMap, step (5) of Fig.~\ref{fig:methodoverview}. In this case, the
destination is the IP address of the server rather than a hostname,
as the DNS resolution was already done from the same network in \S~\ref{sec:source-based}.
We analyze the latency to the last hop, subtracting the latency from the first hop
as before.
We launch 598 measurements; we only measure each destination IP
once from each AS-country pair--with the AS being the 
same as that from the source-based measurements and the country 
being that inferred by IPMap for that IP--regardless of how many times the destination IP 
appears in the source-based measurements. 

We exclude 19 measurements, step (6) of Fig.~\ref{fig:methodoverview}, due to unresponsiveness of either the last hop or the 
RIPE Atlas probe, and 57 due to insufficient
granularity in the RIPE IPmap inference (or the probe's location) to compute geodesic distance. 
Of the remaining 522 measurements, in 385 cases we rely on the difference in latency between 
the last and first hops, and use the last hop latency in all others. Of these,
130 exhibit higher latency to the first hop than the last, a contradiction that 
might be caused by additional (home) router delays due to the generation of an ICMP response,
compared to forwarding an incoming ICMP message from another device. Unlike in
\S~\ref{sec:source-based}, we keep these measurements here as by now we have
at least three pieces of evidence that the server is in a non-adequate country,
decreasing the likelihood that the server is located in the EU (recall that our goal
is to remove as many false positives as possible, as that would erroneously indicate a potential GDPR violation).
In 7 additional cases, the
latency to the first hop is not available (router did not respond to ICMP request).

We then infer whether this latency is consistent with the geodesic distance between 
the RIPE Atlas probe and the destination IP as inferred by RIPE IPmap. 
To account for the Internet's non-geodesic routing due to 
physical constraints, such as the speed of light in fiber being $2c/3$~\cite{10.1145/3402413.3402415}, 
or infrastructure delays, such as queue buildups on routers, our
upper bound for observed speed is $4c/9$~\cite{10.1145/1177080.1177090} or approx. $133 km/ms$; 
this is a more conservative 
threshold than the frequently used $2c/3$. 
If the speed inferred from the 
traceroute round-trip travel time and the geodesic distance between the endpoints 
is higher than $4c/9$, we discard the measurement,
which happens in 89 instances. We then have 433 measurements remain that target servers 
still inferred to be in non-adequate countries.

\subsection{Reverse DNS Lookups}

As a final piece of evidence in our server geolocation methods, we inspect
reverse DNS (rDNS) records of each traceroute's last hop
(reported by RIPE Atlas), step (7) of Fig.~\ref{fig:methodoverview}. Hostnames obtained from rDNS are often, 
but not always, useful in geolocating IP infrastructure~\cite{geodns}, 
which is why this is the last step in our analysis. 

Of the 433 measurements from the last subsection, 255 include
hostnames that confirm the server's country inferred in previous steps
(206 of these refer to servers in the US).
For instance, hostname \textbf{unn}-138-199-8-197.datapacket.com most likely 
refers to IP infrastructure near 
Ranong Airport (IATA code: UNN) in Thailand, which is the same country as inferred 
by IPMap for the corresponding server's IP address.
Given the diversity in operational practices to assign hostnames to IP infrastructure, 
it is not trivial to automatically
infer geographic hints to determine where the referenced infrastructure
is located; our re-implementation of recent work seemed to miss some geographic hints in hostnames~\cite{geodns}, 
which is why we manually inspect all the hostnames in this step, an 
effort that is supported by the data's manageable scale.

The rDNS records for a further 13 traceroutes suggest that the server is located in 
a different non-adequate country than that inferred by RIPE IPMap. In these cases,
we reassign the IP to the non-adequate country inferred from rDNS 
(which tends to be more accurate than latency-based inferences). 
Furthermore, the hostnames for 37 measurements suggest that the servers are located in either the EU
itself, or an adequate third-country. Nearly all of these (31) refer to AWS infrastructure 
that seems to be located in Canada but were erroneously inferred by IPMap to be in the US,
\textit{e.g.}, ec2-99-79-143-255.\textbf{ca-central-1}.compute.amazonaws.com.
We exclude these 37 IPs from further processing, as these servers are unlikely to be located in a non-adequate 
country (recall that Canada is an adequate country~\cite{Adequacy38:online}).  

Finally, 45 measurements do not return a hostname with the rDNS lookup, 
and another 83 do not seem to encode geographic locations.  We keep these servers' location 
inference unchanged from previous steps. 

\subsection{Final Sample of Non-Adequate Servers}
\label{subsec:finalsample}
We are left with 396 measurements to 247 server IP addresses where all available
evidence suggests that the server responding to EU requests is located in a non-adequate country.
These potentially non-adequate servers are present in our data in 1,233 instances,
as the same server may be contacted by multiple initial sites in each ASCP. 
Table~\ref{tab:servergeo} shows all methods we used to investigate the geolocation of servers.
We include the latency distribution for both the source- and destination-based measurements in the
appendix.
\begin{table}
  \caption{Sever Geolocation. *No hostname/no geohint. **RIPE IPMap inferences.}
  \label{tab:servergeo}
  \begin{tabular}{ccccc}
    \toprule
    \textbf{Method} & \textbf{Probes} & \textbf{Unres-} & \textbf{Adequate} & \textbf{Non-}\\
     &  & \textbf{ponsive} &  & \textbf{adequate}\\
    \toprule
    Source& 9905**  & 158 &  8488 & 1259 (598 IPs)\\
    traceroutes &   &  &   & \\
    \midrule
    Destination &598 & 76 & 89 & 433\\
    traceroutes &   &  &   & \\
    \midrule
    Reverse DNS & 433 & 45/83* & 37 & 396 \\

  \bottomrule
\end{tabular}
\end{table}


