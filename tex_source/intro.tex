\section{Introduction}

Regulators around the world have taken various approaches at mitigating potential harms
resulting from unfettered collection of user data at a large scale by Internet companies, 
usually for the purpose of targeted
advertising. Data localization---broadly defined as the principle that 
requires user data to be stored in the jurisdiction where the user is physically located---is 
one such 
regulatory approach and a key component of the EU's privacy landmark regulation enacted in 2018: 
The General Data Protection Regulation (GDPR). 
A central intent behind (GDPR) data localization is to prevent transfers of user 
data to jurisdictions with weaker privacy protections, as such transfers expose 
(EU) users to unlawful commercial surveillance of user data by both commercial 
advertisers and foreign government agencies, 
as was argued in the landmark \textit{Schrems} decisions \cite{schremsi,schremsii}.  


In this paper, we explore the question of whether content providers serving users in Europe 
comply with data localization regulations by locating their servers in the 
EU (or a small number of third countries that the EU regulators have approved 
data transfers to~\cite{Adequacy38:online}). In answering this question, we address a key technical 
challenge. Since data localization is based on physical boundaries, three types 
of information are simultaneously needed to audit compliance with data localization, 
and no existing platform nor methodology can procure all three: 
(\textit{i}) Which domains are most frequently visited by European users, including many additional 
domains fetched automatically 
by user-loaded websites. (\textit{ii}) The location in the network of all relevant 
domains, that is, their IP addresses. (\textit{iii}) The physical location of the servers from 
which these domains are loaded. While (\textit{iii}) is the key information needed to audit 
data localization practices, in practical terms it is difficult to obtain it directly from (\textit{i}) 
domain names; 
rather, it is easier (though still challenging ~\cite{10.1145/3402413.3402415}) to obtain the physical location of IP addresses, 
a necessary resource to connect to the Internet an actual server referenced by each domain.

Our auditing framework on data localization compliance is built with 
three components, each obtaining the necessary information described above. 
First, we identify all popular domains in each EU country using queries from BrightData~\cite{BrightData}, 
a proxy service that allows us to run a headless browser and therefore fetch complete web pages
from EU-based devices. 
Second, we launch active measurements from RIPE Atlas~\cite{RipeAtlasInfo} 
to identify the IP addresses serving each domain. Third, we use RIPE IPMap,~\cite{10.1145/3402413.3402415}) 
a database that translates IP addresses to a 
physical location (IP geolocation),
to identify IP addresses 
that are potentially located in countries 
outside the EU. We confirm a subset of these as likely 
GDPR violations using active measurements and speed-of-light constraints.
We validate our method using servers with known locations,
finding no false positives.

We find that the vast majority of servers responding to requests from EU users are 
located in the EU or adequate third countries. %
Neighbors of the EU, Russia and Turkey, are the most commonly observed destinations 
in non-adequate countries, besides the US. These servers
primarily serve requests in Finland and Romania, respectively.
We find smaller numbers of servers and known trackers in other non-adequate countries
further away from the EU, primarily in Asia.
News websites are the leading cause of these potential GDPR violations,
as they are the most common type of site
that loads trackers in non-adequate countries.
We find evidence of tracking activity taking place in a non-adequate 
country using cookies as well.
We also find significant differences across EU regions: Southern and Eastern EU
countries are more likely to be served content by known trackers located in 
non-adequate countries when compared to Western and Northern EU countries.

Combined, these findings suggest that Internet companies are likely inclined to host content 
in close proximity of users, which leads to by-default compliance with data localization 
regulations; however, there are exceptions where rigid technical constraints
may force Internet operators to serve EU users from non-compliant jurisdictions. 
In these circumstances, the imposition of physical constraints on the Internet infrastructure
is likely to be more challenging than currently recognized by both Internet operators and regulators alike,
which is especially relevant as proposed regulations on data localization emerge in more 
countries~\cite{Indiasd23:online,Dataloca93:online}. 

\section{Policy Background}
\label{sec:policy}
\if 0
 I imagine this would focus on 
 
 (1) the language of the GDPR requirement, 
 
 (2) any relevant process/case law/interpetation/legal guidelines for deeming a country to be adequate, and 
 
 (3) examples of countries that are/are not authorized for cross-border transfers, in addition to discussing the specific US 
 \fi
 

Data transfers to jurisdictions with weaker privacy protections
expose users in the initial jurisdiction to potential harms
such as leaks of personal information to foreign governments
or advertisers. 
Despite these potential harms, little is known about whether or how 
Internet companies that serve EU users operate their 
infrastructure to comply with this provision of the GDPR (Chapter V). 

Specifically, the GDPR in Article 45 mentions that 
``A transfer of personal data to a third country or an international organisation may take place where the [European] 
Commission has decided that the third country, a territory 
or one or more specified sectors within that third country, or the 
international organisation in question ensures an adequate level of protection.''~\cite{gdpreu}
Generally speaking, the article refers to several principles that must be followed 
in such countries receiving data transfers from EU persons, 
such as ``elements like rule of law, respect for human rights and 
fundamental freedoms, as well as whether or not data subjects' 
rights are effective and enforceable, the existence and effective 
functioning of an independent data protection authority in the non-EEA 
country and the international commitments the country or 
international organisation has entered into.''~\cite{edpb}
These adequacy decisions are published on a European Commission (EC) 
website~\cite{Adequacy38:online} and the decisions must be reviewed 
and renewed periodically by law. At the time of writing, the EC has 
determined that transfers between the EU and other European Economic Area countries 
(Norway, Liechtenstein, and Iceland) are treated as intra-EU 
transfers without needing any further safeguard. Also, according to that 
same website, the EC ``has so far recognised Andorra, Argentina, Canada 
(commercial organisations), Faroe Islands, Guernsey, Israel, Isle of Man, 
Japan, Jersey, New Zealand, Republic of Korea, Switzerland, the United 
Kingdom under the GDPR and the LED, the United States (commercial 
organisations participating in the EU-US Data Privacy Framework) 
and Uruguay as providing adequate protection.'' 

Any countries not covered by the above list are potentially inadequate, 
and any international data transfers from the EU to them would violate Article 
45 if that country is deemed inadequate by EU courts, 
\textit{e.g.}, the Court of Justice of the European Union (CJEU) or the EC. 
Even countries previously determined to be adequate may face a reversal of that decision. 
For example, in 2020, the CJEU determined in \textit{Schrems II}~\cite{schremsii}  
that ``As a consequence of such a degree of interference with the fundamental rights of persons whose
data are transferred to [the United States, where national security 
laws limit data protections], the Court declared the Privacy Shield adequacy
Decision invalid.''~\cite{edpb2}
We note that data transfers to the US were not broadly authorized when we collected our data in Aug-Sept 2022,
as the EU-US Data Privacy Framework was not adopted until December of that year ~\cite{EUUSData30:online}.
The EU-US DPF was adopted in response to the \textit{Schrems II} decision and determined adequate by the EC in July 2023.~\cite{ecreport}


Given regular instances of noncompliance resulting in EU-imposed penalties on Internet 
companies~\cite{Asitsdat63:online,30Bigges7:online}, 
it is clear that content providers are not universally
complying with these GDPR provisions. 
Besides these anecdotal instances, 
and studies with limited relevance (see
\S~\ref{sec:related}), 
there is currently no EU-wide study of whether data localization 
principles are being complied with in practice by content providers. 

Indeed, privacy advocates have primarily relied on the \textit{possibility} 
that data may be transferred to a jurisdiction with weaker privacy protections, 
especially transatlantic transfers which might result in exposure to US government surveillance. 
While these claims have successfully 
moved the needle on public policy (holding large tech companies such as Google accountable for transatlantic
data transfers~\cite{DSBAustr4:online}) through the European courts, 
they are based on analyses of the tech companies' 
own privacy policies and manual, surface-level analysis of HTML~\cite{101Compl42:online}; 
thus, these methodologies are neither aimed at nor adequate to audit compliance of 
data localization by content providers at EU scale. 

In this paper, we focus on data transfers to countries not 
determined to be adequate at the time of measurement (see definition of adequate above). 
We refer to such countries as \emph{non-adequate}.
Thus, our study identifies \textit{potential} GDPR violations, that is the existence of a tracking server in a non-adequate country.





