\section{Server Geolocation Validation}
\label{sec:testbed}

To validate our server geolocation method, we build an experiment 
based on two sets of server IPs with known location.
Each of these sets is based either in a non-adequate country, the 
US,~\footnote{At the time of our main data collection. The validation experiment 
was conducted in Feb. 2025, when the US had already received an adequacy decision.~\cite{latham}}
or an adequate country, an EU member state.
To conduct this validation experiment, we followed the four steps 
of the methodology defined in \S~\ref{sec:servergeolocation} to perform 
server IP geolocation: RIPE IPmap geolocation, source-based constraints, 
destination-based constraints, and reverse DNS (rDNS).

The results of this experiment are shown in Tab.~\ref{tab:testbedgeo}.
In the following subsections, we describe this validation experiment.
We also infer rates of true/false positives and precision of our method.
Note that, as discussed in the limitations section (\S~\ref{sec:falsepos}), the 
rate of false positives might be different in our main results.




\subsection{Server Testbed in the US}
We have administrative access 
to a US testbed, CloudLab ~\cite{CloudLab}, and thus to ground truth mapping between server IP and country of operation. 
We conducted experiments using 200 such servers from a testbed that are
distributed across two different locations (L1 and L2), both in the US.
These servers are the destination IPs for this experiment. 
Further, we selected five source countries from 
the European Union (EU). We included the largest EU country, France (by land area), 
and four additional countries chosen randomly to represent different regions of the EU: 
Poland from Eastern Europe, Germany from Western Europe, Ireland from Northern Europe, 
and Spain from Southern Europe.
For each source country, we randomly selected 40 server IPs as destinations across 
the two testbed locations (L1 and L2). This results in a total of 200 destination IP-source
country pairs (DSCPs). 

The upper half of Tab.~\ref{tab:testbedgeo} shows the results of this
experiment. In the rest of this paragraph, we describe these 
findings in more detail. 
We found that RIPE IPmap did not provide country-level geolocation 
for three server IPs. %
In the source-based traceroutes, five server IPs failed to respond, and 20 did not meet the 90\% 
latency threshold described in \S~\ref{sec:source-based}. 
We further excluded two server IPs
using destination-based constraints. %
Finally, when we analyzed rDNS data, 
three of these servers had clues indicating they were not located in the city identified by RIPE IPmap,
but they were still located in the US. As our method is concerned with
country-level geolocation, we keep the server geolocation unchnaged from previous steps.
In summary, our final dataset correctly geolocated 170 DSCPs 
as being in the US,
and discarded 30 DSCPs due to the constraints described above.








\vspace{-10mm}
\subsection{AWS Servers in the EU}

To validate our method for adequate server IPs, 
we conducted an experiment using 1,000 IPs in the EU 
advertised by Amazon Web Services~\cite{aws_ip_ranges}. 
We selected these IPs by intersecting the AWS-published IP ranges
with the ISI IP Hitlist,~\cite{ant_hitlist} which estimates the likelihood
that a server will respond to network measurements.
The IPs we select have a score of 99 on the Hitlist. 
We take this step in an attempt to maximize response rates to our measurements 
since, unlike on the US testbed, 
we have no direct knowledge of whether the AWS IPs are currently
both routed and in use. This is also why we increase the 
number of server IPs in the EU relative to the US experiments in the previous
subsection.

Similarly to our treatment of the US testbed, we divide the server IPs 
into five groups corresponding to the same five 
EU source countries. 
The lower half of Tab.~\ref{tab:testbedgeo} shows the results of this experiment.

From these 1,000 DSCPs, we find that RIPE IPMap 
does not assign a country in 36 cases, and erroneously geolocates 
270 DSCPs outside of the EU in non-adequate countries: 
one in Pakistan, the rest in the US. 
After conducting both our source- and destination-based measurements, 
and applying speed of light constraints, these 270 DSCPs are all discarded.
At this stage, 27 DSCPs are correctly labeled as being in an (adequate)
EU country.
rDNS confirms this assertion in all 27 cases.



\begin{table}
  \caption{DSCP Geolocation for server validation experiment. 
ST/DT: Source/Destination Traceroutes. 
TB: Server Testbed. *No hostname/no geohint.}
  \label{tab:testbedgeo}
  \begin{tabular}{ccccc}
    \toprule
    \textbf{Method} & \textbf{DSCPs} & \textbf{Unres-} & \textbf{Adequate} & \textbf{Non-}\\
     &  & \textbf{ponsive} &  & \textbf{adequate} \\
    \toprule
    ST: US TB & 197  & 5 & 0 & 172 \\
    DT: US TB & 172 & 0 & 0& 170 \\
    rDNS: US TB & 170 & 99/0* & 0 & 170 \\
    \midrule
    ST: AWS & 964  & 910 & 50  & 0 \\
    DT: AWS & 50  & 21 & 27  & 0 \\
    rDNS: AWS & 27  & 0/0* & 27  & 0 \\
  \bottomrule
\end{tabular}
\end{table}


\subsection{True/False Positives and Negatives, Precision}
In this subsection, we describe the rates of
true/false positives and negatives,
along with precision, that result
from our validation experiments.

\textbf{False positives and true negatives}. 
The outcome of the AWS experiment is either a false positive,
where a server 
in an adequate country is incorrectly inferred as being in a
non-adequate country; or a true negative,
where a server in an adequate country is correctly labeled as being so.

Given the results of the AWS validation experiment, 
we infer \textit{no false positives}.
This is because all servers initially labeled as being outside the 
EU by RIPE IPMap were correctly discarded---assigned a negative label---by 
subsequent steps in our method. 
Note that our source- and destination-based filters 
aggressively discard DSCPs because they are
either unresponsive or the latency fails our speed of light constraints.
Thus, after all filters are applied, 
the \textit{true negative} rate is 100\%.

In aggregate, these results suggest that the coverage of our method is limited,
because it excludes servers that may actually be located 
in a non-adequate country. 
Simultaneously, our method reduces the likelihood of false positives. 
Thus, in this experiment, our method is working as intended.


\textbf{True positives and false negatives}. Recall that the US was a 
non-adequate destination at the time of our main data collection,
so all servers in the US testbed
have a non-adequate country as their ground truth location; %
thus, all DSCPs in the US testbed validation experiment are either
correctly allocated to the US, a true positive, or discarded, a false negative.
Our method correctly identified 170 DSCPs as being in the US,
of 200 DSCPs known to be in the US.
Therefore, the \textit{true positive} rate of our method in this experiment is 85\%,
and the \textit{false negative} rate is 15\%.

\textbf{Precision}. From the aforementioned results,
in particular given the absence of false positives,
the inferred \textit{precision} of our method 
in the validation experiments is 1.0.

